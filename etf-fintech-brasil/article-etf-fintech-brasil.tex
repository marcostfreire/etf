\documentclass[12pt]{article}
\usepackage[utf8]{inputenc}
\usepackage[T1]{fontenc}
\usepackage{lmodern}
\usepackage{amsmath,amssymb}
\usepackage{geometry}
\usepackage{hyperref}
\usepackage{graphicx} % Para inserir figuras (placeholders)

\geometry{a4paper, margin=2.5cm}

\begin{document}

\section*{Dados do Mercado Fintech 📊}

O Brasil consolidou-se como o principal hub de fintechs na América Latina, abrigando cerca de 30\% a 40\% das startups financeiras da regiã (O novo cenário das fintechs em 2024, de olho nos mercados B2B – ANBIMA) (Fintechs brasileiras atraíram quase R\$ 60 bi em investimentos em 10 anos, diz estudo sobre as startups que desafiam os bancos | Startups | Época NEGÓCIOS)】. Em 2023, o país já contava com aproximadamente 1.500 fintechs ativas, posicionando-se em 1º na América Latina e 7º globalmente em número de fintech (O novo cenário das fintechs em 2024, de olho nos mercados B2B – ANBIMA) (The Fintech Ecosystem in Brazil)】. Esse ecossistema inclui tanto startups locais quanto operações estrangeiras no país – um estudo mapeou 1505 fintechs no Brasil (447 consideradas “locais”) distribuídas em segmentos como \textit{pagamentos, crédito, investimentos, regtech, seguro e bancos digitais} (The Fintech Ecosystem in Brazil) (The Fintech Ecosystem in Brazil)】. Fintechs já representam o maior e mais financiado setor de startups no Brasil, respondendo por \textasciitilde1.476 das 13.365 startups ativas em 2023 (11\% do total (O novo cenário das fintechs em 2024, de olho nos mercados B2B – ANBIMA)】. Em escala global, o movimento também é robusto: estima-se que haja quase 30 mil fintechs no mundo em 2024, e a América Latina é o 4º maior mercado em quantidade de fintechs (cerca de 4,7 mil (Fintech: vantagens do Low Code e BPMS para o setor) (Infographic: LatAm Fintech Landscape, 2023)】.

\textbf{Tendências:} Nos últimos anos, o setor fintech brasileiro tem passado por mudanças de foco e amadurecimento. Inicialmente dominadas por soluções B2C, cada vez mais fintechs migram para modelos B2B em busca de rentabilidade – mais da metade (56\%) das fintechs brasileiras já atendem apenas empresas, ante 40\% no ano anterio (O novo cenário das fintechs em 2024, de olho nos mercados B2B – ANBIMA) (O novo cenário das fintechs em 2024, de olho nos mercados B2B – ANBIMA)】. Áreas como pagamentos digitais (por exemplo, carteiras eletrônicas e Pix), crédito online, bancos digitais/neobanks e infraestrutura tecnológica financeira lideram a inovação. O Brasil tem se destacado em pagamentos instantâneos: o Pix, lançado pelo Banco Central em 2020, revolucionou o sistema de pagamentos. Em menos de dois anos, o Pix já contava com *139 milhões de usuários (75\% da população) (O Surpreendente Cenário Favorável Para as Fintechs no Brasil | Andreessen Horowitz)】. Em 2023, o Pix atingiu 42 bilhões de transações, crescendo 75\% vs. 2022 e tornou-se o meio de pagamento mais usado no paí (FEBRABAN - Notícias) (FEBRABAN - Notícias)】. Essa adoção maciça fez com que o Pix representasse 39\% de todas as transações de pagamento em 2023, encostando nos cartões (41\%) e superando boletos, TED/DOC e outros meios tradicionai (Pix encosta nos cartões e bate 39\% dos pagamentos - Finsiders Brasil)】. Tendências globais como “embedded finance” (finanças embutidas) e Open Finance também ganham força localmente – projeta-se que finanças embutidas movimente US\$ 320 bilhões globalmente até 203 (Relatório aponta o "novo" ecossistema global de fintechs - Finsiders Brasil)】, e o Brasil, com seu ecossistema digital dinâmico, deve contribuir significativamente para esse crescimento. Além disso, o uso de Inteligência Artificial generativa (IA) está no radar das fintechs e bancos para melhorar atendimento, análise de dados e marketin (Relatório aponta o "novo" ecossistema global de fintechs - Finsiders Brasil)】.

\textbf{Regulação e Impacto Econômico:} O ambiente regulatório brasileiro tornou-se um diferencial pró-inovação no setor financeir (O Surpreendente Cenário Favorável Para as Fintechs no Brasil | Andreessen Horowitz)】. Na última década, o Banco Central implementou uma agenda inovadora: criou modalidades de instituições de pagamento (2013), regulou fintechs de crédito (Res. 4.656/2018) permitindo peer-to-peer lending e Sociedades de Crédito Diret (Análise: O impacto econômico das fintechs brasileiras)】, lançou o Pix (2020) e coordenou a implementação do Open Banking/Finance (a partir de 2021). Essas medidas aumentaram a concorrência e a inclusão financeira. A taxa de bancarização saltou de 57\% para 86\% da população com acesso a serviços financeiros, trazendo \textasciitilde75 milhões de brasileiros ao sistema bancário nos últimos ano (O Surpreendente Cenário Favorável Para as Fintechs no Brasil | Andreessen Horowitz)】 – um avanço impressionante atribuído em boa parte às fintechs e aos novos bancos digitais. O efeito competitivo começa a aparecer: após décadas de concentração, os cinco grandes bancos reduziram levemente sua fatia de crédito de 81,8\% em 2020 para 81,4\% em 202 (Análise: O impacto econômico das fintechs brasileiras) (Análise: O impacto econômico das fintechs brasileiras)】. Incluindo fintechs e novas instituições não bancárias, a concentração de crédito cai para \textasciitilde68\%, indicando espaço sendo conquistado pelos novos entrante (Análise: O impacto econômico das fintechs brasileiras)】. As fintechs de crédito, em particular, multiplicaram a oferta de empréstimos pessoais e para PMEs, contribuindo para reduzir spreads de juros segundo o FM (The Fintech Ecosystem in Brazil)】.

No âmbito macroeconômico, as fintechs impulsionam a eficiência e inclusão, democratizando serviços financeiros e reduzindo custos. Por exemplo, 5,7 milhões de brasileiros obtiveram acesso a crédito através do Nubank (cartão de crédito) em apenas um an (como o Nubank impacta diariamente 70 milhões de brasileiros)】. Estão também ajudando a movimentar a economia: somente o Pix movimentou R\$ 17,2 trilhões em 2023 e já R\$ 26,5 trilhões em 2024 (alta de 54\% no 1º semestre (Em novo ano recorde, Pix movimenta R\$ 26,5 trilhões em 2024)】. Além disso, fintechs bem-sucedidas tornaram-se grandes empregadoras e pagadoras de impostos, enquanto a digitalização forçada dos incumbentes gera ganhos sistêmicos. O Brasil lidera a captação de recursos de fintechs na região, tendo atraído US\$ 10,4 bilhões (R\$ 60 bi) em investimentos entre 2014 e 2023, o que corresponde a 66\% de todo o aporte em fintech na América Latina no períod (Fintechs brasileiras receberam 66\% dos aportes de LatAm em 10 anos - Startups) (Fintechs brasileiras receberam 66\% dos aportes de LatAm em 10 anos - Startups)】. Esse volume de capital trouxe ao país dezenas de “unicórnios” fintech (startups avaliadas acima de US\$ 1 bi), como Nubank, Ebanx, Creditas, C6 Bank, etc., gerando valor econômico substancial. Vale notar que o Nubank fez IPO em 2021 avaliado em >US\$ 50 bilhões, tornando-se à época o banco mais valioso da América Latin (The Fintech Ecosystem in Brazil)】. Apesar da volatilidade de mercado recente, o setor continua crescendo: a receita das fintechs deve crescer quase 3x mais rápido que a dos bancos tradicionais entre 2023 e 2028, com CAGR \textasciitilde15\% vs. \textasciitilde6\% dos banco (Fintechs deverão crescer dois dígitos ao ano até 2028 | McKinsey)】 – sinal de um impacto econômico cada vez maior à frente. (The Fintech Ecosystem in Brazil)】 

\textbf{Gráfico 1:} Panorama do ecossistema fintech brasileiro em 2023. O Brasil conta com 1505 fintechs mapeadas, distribuídas em segmentos como WealthTech (317), PayTech (423), Lending/Crowd (303), RegTech/SupTech (122) e Neobanks (27). A maioria adota modelos de negócio B2B (86\%) e/ou B2C (76\%), muitas atuando em ambos os mercados. Tecnologias como IA, Blockchain e Big Data já são empregadas por cerca de 10\% das fintechs cada. (Fonte: Finnovating / Fintech Global Vision 202 (The Fintech Ecosystem in Brazil) (The Fintech Ecosystem in Brazil)】)*

% --- Placeholder para o Gráfico 1 ---
\begin{figure}[!h]
\centering
\includegraphics[width=0.75\textwidth]{grafico1_placeholder.png}
\caption{\textbf{Gráfico 1 (Placeholder).} Panorama do ecossistema fintech brasileiro em 2023.}
\label{fig:graph1}
\end{figure}

\section*{Crescimento do Setor e Métricas para ETFs 🚀}

\textbf{Investimentos em Fintechs:} O fluxo de capital de risco para fintechs teve crescimento explosivo na última década, culminando em um pico em 2021, quando as fintechs brasileiras captaram cerca de US\$ 5,7 bilhões em 363 rodadas de investimento – um recorde históri (Fintechs brasileiras receberam 66\% dos aportes de LatAm em 10 anos - Startups)7】. Esse boom refletiu o apetite de investidores globais em meio a juros baixos e casos de sucesso locais (várias fintechs tornaram-se unicórnios). Em 2022, porém, houve uma correção significativa: os investimentos em fintech no Brasil caíram \textasciitilde44\%, para cerca de US\$ 2,3 bilhões, acompanhando a retração global diante de inflação e alta de jur (Análise: O impacto econômico das fintechs brasileiras) (Análise: O impacto econômico das fintechs brasileiras)5】. Ainda assim, mesmo em 2022 as fintechs se mantiveram como o setor líder em aportes no pa (Análise: O impacto econômico das fintechs brasileiras)5】, superando o volume de 2020. Já em 2023 e 2024, observa-se uma retomada: somente no 1º semestre de 2024, os investimentos em fintechs na AL cresceram 75\% vs. H1 20 (Fintechs brasileiras receberam 66\% dos aportes de LatAm em 10 anos - Startups)9】. Em uma década (2014-2023), as fintechs brasileiras receberam mais de R\$ 59 bilhões em 1.034 deals, concentrando **66\%** do volume de venture capital do segmento na LatAm (Fintechs brasileiras atraíram quase R\$ 60 bi em investimentos em 10 anos, diz estudo sobre as startups que desafiam os bancos | Startups | Época NEGÓCIOS) (Fintechs brasileiras receberam 66\% dos aportes de LatAm em 10 anos - Startups)7】. Esses números evidenciam o vigor do setor e o interesse contínuo dos investidores, apesar de ciclos de alta e baixa.

\textbf{Desempenho do setor e ETFs comparáveis:} No mercado público de ações, as fintechs apresentaram forte crescimento de valor até 2021, seguido de volatilidade. Entre 2019 e 2023, as fintechs de capital aberto globalmente dobraram de valor de mercado, atingindo US\$ 550 (Fintechs deverão crescer dois dígitos ao ano até 2028 | McKinsey) (Fintechs deverão crescer dois dígitos ao ano até 2028 | McKinsey)7】, mas sofreram correções conforme investidores passaram a exigir lucratividade. Esse movimento abriu oportunidades para ETFs temáticos de fintech, que permitem investir de forma diversificada no setor. Um exemplo global é o ETF Global X FinTech (FINX), listado nos EUA, cujo índice acompanha \textasciitilde64 empresas de tecnologia financeira. Desde seu início em 2016 até início de 2025, o FINX acumulou **+118\%** de valorização (FinTech ETF (FINX))6】 – mais que dobrando o capital – refletindo o crescimento da tese fintech. Nos últimos 2 anos (24 meses), o ETF subiu cerca de **42,7\%** (FinTech ETF (FINX))9】, superando muitos índices tradicionais, embora com volatilidade significativa. Em 2022, por exemplo, o FINX chegou a cair forte junto com as techs, mas em 2023 recuperou com altas expressivas (ex.: +17\% em Nov/202 (FinTech ETF (FINX)) (FinTech ETF (FINX))2】. Isso demonstra o caráter cíclico porém robusto do setor – períodos de correção podem ser seguidos por rally de crescimento alinhado às tendências de longo prazo.

No Brasil, ainda não há um ETF 100\% focado em fintechs locais, mas as empresas do setor listadas (no Brasil ou exterior) também tiveram desempenhos mistos recentemente. Muitas fintechs abriram capital entre 2018 e 2021 (como PagSeguro, Stone, Banco Inter, XP Inc, e Nubank em 2021) surfando altas avaliações, mas depois viram seus preços ajustarem. Ainda assim, os fundamentos vêm melhorando: no 1º semestre de 2023, 7 das 8 principais fintechs brasileiras registraram lucro líquido positivo, um salto em relação a 20 (Sete de oito fintechs brasileiras registram lucro no primeiro semestre -)2】. O grupo (Nubank, Inter, Stone, PagSeguro, XP, PicPay, etc.) lucrou **R\$ 4,96 bi em H1 2023**, crescimento de 38\% vs. H1 2022 (Sete de oito fintechs brasileiras registram lucro no primeiro semestre -)2】 – sinal de maturidade financeira. Isso favorece a tese de um ETF fintech, pois investidores tendem a valorizar setores em que as empresas já demonstram sustentabilidade.

\textbf{Métricas-chave de ETFs de Fintech:} Para avaliar um ETF temático como esse, algumas métricas e indicadores financeiros são relevantes:
\begin{itemize}
\item \textbf{Crescimento de Receita e Lucro das empresas:} Fintechs ainda estão em trajetória de alto crescimento. Globalmente, as receitas do setor vêm crescendo \textasciitilde14\% ao ano (2021-202 (Relatório aponta o "novo" ecossistema global de fintechs - Finsiders Brasil) (Relatório aponta o "novo" ecossistema global de fintechs - Finsiders Brasil)7】 e projetam CAGR de \textasciitilde15\% até 20 (Fintechs deverão crescer dois dígitos ao ano até 2028 | McKinsey)7】 – bem acima da média de mercado. Espera-se que um ETF fintech capture esse crescimento superior, resultando em aumento de valor para cotistas.

\item \textbf{Valuation (Preço/Lucro, Preço/Vendas):} Como muitas fintechs abriram capital recentemente, várias ainda negociam com P/L alto ou lucro reduzido. Porém, após a correção recente, as avaliações ficaram mais atrativas – por exemplo, o múltiplo médio Price/Sales das fintechs públicas globais caiu de \textasciitilde20x em 2021 para \textasciitilde4x em 20 (Relatório aponta o "novo" ecossistema global de fintechs - Finsiders Brasil)7】. No caso do ETF FINX, o **P/L forward** médio está em torno de 22,6x para 2025 (FinTech ETF (FINX))4】, indicando expectativa de lucros crescentes. Tais indicadores sugerem que o setor, embora ainda “growth”, não está mais tão sobreavaliado como no auge, o que é positivo para novos investidores.

\item \textbf{Ativos sob Gestão (AUM) e Liquidez:} O sucesso de um ETF se mede também pelo volume captado e liquidez das cotas. O FINX global, por exemplo, possui \textasciitilde US\$ 287 milhões em A (FinTech ETF (FINX))7】. No Brasil, a Investo (gestora especializada em ETFs) já alcançou R\$ 200 milhões em AUM em \textasciitilde1 ano de operação com 12 ET (Investo capta R\$ 40 milhões para popularizar ETFs no Brasil - Startups)8】, e espera chegar a R\$ 1 bi com 20 ET (Investo capta R\$ 40 milhões para popularizar ETFs no Brasil - Startups)7】 – mostrando apetite dos investidores locais. Um novo ETF Fintech Brasil poderia facilmente buscar dezenas de milhões de reais em captação inicial e crescer conforme o tema ganhasse tração.

\item \textbf{Desempenho vs. Benchmarks:} Investidores avaliarão como o ETF se compara a índices amplos (Ibovespa, S\&P500) ou setor financeiro tradicional. A expectativa é que, no longo prazo, fintech supere bancos tradicionais em crescimento – a McKinsey estima fintechs crescendo 3x mais rápido que banc (Fintechs deverão crescer dois dígitos ao ano até 2028 | McKinsey)7】 – potencialmente entregando \textit{alfa} (retorno acima do mercado). Monitorar o \textit{tracking error} em relação a um índice de referência fintech e \textit{sharpe ratio} (retorno ajustado ao risco) serão importantes para provar o valor do ETF.

\item \textbf{Volatilidade (Beta):} Setores de tecnologia tendem a oscilar mais. O FINX possui $\beta \sim1,65$ vs S\&P500 e desvio-padrão de \textasciitilde33\% ao a (FinTech ETF (FINX))2】, indicando maior risco porém maior potencial de retorno. Investidores devem estar cientes dessa volatilidade. No ETF proposto, diversificação interna ajudará a suavizar riscos idiossincráticos, mas provavelmente o beta será >1 (possivelmente \textasciitilde1,3–1,5) dado o perfil arrojado.

\item \textbf{Outras métricas:} \textit{Expense ratio} (taxa de administração) competitiva – ETFs Investo giram em torno de 0,5\% a 0,7\% a.a. No caso de um ETF fintech, justificar essa taxa com a curadoria especializada é essencial. Também, conferir liquidez diária das cotas (volume negociado) – por isso é crucial compor com ações líquidas.
\end{itemize}

Em suma, o setor exibe forte crescimento de fundamentos e certa estabilização nos múltiplos, tornando o \textit{timing} atrativo para lançar um ETF de fintechs. A performance de veículos semelhantes e os indicadores sinalizam um potencial de sucesso ao replicar essa tese no mercado brasileiro.

% --- Placeholder para possível gráfico de tendência de investimentos ---
\begin{figure}[!h]
\centering
\includegraphics[width=0.75\textwidth]{grafico_tendencia_investimentos_placeholder.png}
\caption{\textbf{Gráfico (Placeholder).} Tendência de investimentos em Fintech ao longo dos anos.}
\label{fig:trendInvest}
\end{figure}

\section*{Composição do ETF 🏦💳📱}

A proposta de composição do ETF Fintech Brasil deve equilibrar exposição às empresas mais promissoras do setor com critérios rigorosos de liquidez e qualidade. A seleção focará em companhias de capital aberto (no Brasil ou exterior via BDR) com atuação destacada em fintech, priorizando aquelas com maior volume de negociação, bom potencial de crescimento (“alavancagem” tecnológica) e valuations razoáveis. Segue uma lista de potenciais componentes-chave do ETF e sua justificativa:

\begin{itemize}
\item \textbf{Nu Holdings (Nubank) – Neobank.} Nubank (NYSE: NU, BDR: NUBR33) é referência global em bancos digitais, atendendo mais de 100 milhões de clientes no Brasil (3º maior banco do país em número de clie (Nubank se torna o 3º maior banco do Brasil em número de clientes)-L47】. Com crescimento vertiginoso e operações também no México e Colômbia, a Nubank combina escala e inovação. A empresa atingiu \textit{breakeven} e apresentou forte lucro ajustado em 2023, o que melhora seu valuation. Por sua liquidez elevada (ações listadas nos EUA) e representatividade do setor, Nubank seria possivelmente a maior posição do ETF.

\item \textbf{PagSeguro (PagBank) – Pagamentos \& Digital Banking.} PagSeguro (NYSE: PAGS, B3: PAGS34) é uma fintech brasileira focada em pagamentos digitais (maquininhas, e-commerce) e serviços bancários via PagBank. É rentável e consolidada no segmento de PMEs e autônomos, com milhões de clientes. Sua ação é líquida na NYSE e a empresa vem diversificando receitas (crédito, banking). Representa o dinamismo do setor de pagamentos no Brasil pós-Pix.

\item \textbf{StoneCo – Pagamentos \& Software financeiro.} Stone (NASDAQ: STNE, B3: STOC34) tornou-se uma das maiores adquirentes de cartões do país, competindo com Cielo e Rede, e expandiu para oferta de crédito e software de gestão para lojistas. Após enfrentar desafios em 2021-22, voltou ao lucro em (Sete de oito fintechs brasileiras registram lucro no primeiro semestre -)L132】. Stone tem investidores de peso (Warren Buffett é acion (Top Brazilian Fintech companies trading on US stock exchanges)-L36】, o que atesta confiança em seu modelo. Altamente líquida, é peça-chave para exposição em meios de pagamento.

\item \textbf{Banco Inter (Inter\&Co) – Banco digital \& Marketplace.} Inter (NASDAQ: INTR, B3: BIDI34) foi um dos pioneiros em banco digital no Brasil. Oferece conta grátis, crédito, investimentos e shopping integrado. Conta com \textasciitilde25 milhões de clientes e opera com modelo de \textit{super app}. A empresa já negocia na Nasdaq e vem melhorando seus resultados, tornando-se uma representante importante dos bancos digitais de capital aberto (ao lado do Nubank).

\item \textbf{XP Inc – Investimentos \& WealthTech.} XP (NASDAQ: XP, B3: XPBR31) revolucionou o mercado de investimentos no Brasil ao democratizar o acesso à bolsa e fundos. Hoje é uma plataforma líder com mais de R\$ 1 trilhão sob custódia. Empresa lucrativa e madura, a XP traz ao ETF exposição à vertical de investimentos/\textit{wealth management} digital. Apesar de não ser “fintech” no sentido startup, é uma empresa de tecnologia financeira consolidada e alinhada ao tema.

\item \textbf{MercadoLibre (Mercado Pago) – E-commerce \& Pagamentos.} MercadoLibre (NASDAQ: MELI, BDR: MELY34) é maior plataforma de e-commerce da AL, mas também uma gigante fintech via Mercado Pago (carteira digital e adquirência) e Mercado Crédito. A inclusão de MELI daria ao ETF uma dimensão regional, e exposição indireta a vários mercados. A empresa é extremamente líquida e valiosa (\textasciitilde US\$ 60 bi de market cap). Embora não seja “brasileira” (é argentina), grande parte de suas receitas vem do Brasil, justificando sua presença.

\item \textbf{Visa Inc. / Mastercard Inc. – Infraestrutura de Pagamentos.} Opcionalmente, poderíamos incluir grandes facilitadores globais como Visa (B3: VISA34) ou Mastercard (B3: MSCD34) para conferir liquidez e exposição ao fluxo de pagamentos digital no Brasil. Essas empresas não são fintechs nativas, mas se beneficiam diretamente da digitalização financeira (e estão investindo em open banking, criptografia, etc.). No entanto, sua inclusão deve ser ponderada para não desvirtuar o foco do ETF – possivelmente com peso menor ou apenas se necessário por critérios de liquidez.

\item \textbf{Outras Fintechs Globais Relevantes:} Para diversificar e dar exposição a tendências emergentes, o ETF pode incluir algumas fintechs internacionais de destaque que atuam também no mercado brasileiro ou inspiram o setor:
\begin{itemize}
\item PayPal (PYPL) – pioneira em pagamentos digitais e carteira online, com presença g (FinTech ETF (FINX))L168】. Atuação no Brasil via e-commerce e remessas.
\item Block (SQ) – controladora da Square e CashApp, forte em pagamentos móveis e c (FinTech ETF (FINX))L174】.
\item Adyen (ADYEN) – holandesa de pagamentos omnichannel, atende muitos varejistas no B (FinTech ETF (FINX))L165】.
\item SoFi (SOFI) ou Affirm (AFRM) – fintechs americanas de empréstimos e BNPL, capturando outras vertentes.
\end{itemize}

Essas adições estrangeiras seriam via BDRs e limitadas a alguns nomes, visando incrementar o potencial do ETF sem perder o caráter majoritariamente brasileiro.

\item \textbf{Empresas de Infraestrutura/B2B fintech:} Para refletir o ecossistema completo, poderíamos incluir companhias de tecnologia que dão suporte às instituições financeiras:
\begin{itemize}
\item TOTVS (TOTS3) – líder em software de gestão no Brasil, que também oferece soluções para o setor financeiro (após aquisição da Linx, entrou forte em meios de pagamento para varejo).
\item Sinqia (SQIA3) – provedor especializado em softwares para bancos e fintechs (core bancário, fundos, etc.), diretamente ligado ao crescimento das fintechs.
\item Fiserv (FI) ou FIS (FIS) – empresas internacionais de tecnologia bancária presentes no Brasil, habilitadoras de processamento, core banking e serviços para fin (FinTech ETF (FINX))L167】.
\end{itemize}
\end{itemize}

Tais empresas tendem a ter receitas estáveis e complementaridade com as fintechs puras, conferindo equilíbrio setorial. No ETF, sua participação seria menor (dado menor “pureza” fintech), mas contribui para estabilidade e \textit{dividend yield} potencial.

\textbf{Critérios de Seleção e Ponderação:} A metodologia do ETF seria transparente e baseada em critérios objetivos. Possíveis filtros:
\begin{itemize}
\item \textit{Liquidez:} somente ações com volume médio diário acima de X milhões (garante que o ETF possa montar e desmontar posições sem impacto de mercado). Ex.: Nubank, Stone, XP, etc., todas têm ADTV elevado.
\item \textit{Tamanho:} empresas com valor de mercado mínimo (p. ex. > US\$ 500 milhões), evitando startups muito pequenas/voláteis.
\item \textit{Foco Fintech:} definição clara do universo elegível – empresas cujo core business esteja em tecnologia financeira inovadora (podendo incluir bancos digitais, mas excluindo bancões tradicionais ou processadoras antigas que não inovam). Por exemplo, Itaú ou Cielo ficariam de fora por não serem “emergentes disruptivas”.
\item \textit{Ponderação:} utilizar peso por valor de mercado ajustado (\textit{market-cap weighting}) para dar maior participação às empresas mais valiosas, mas com limitações de concentração. Poderia-se aplicar um \textit{cap} de, digamos, 20\% por empresa para evitar que Nubank (a maior) domine excessivamente. Alternativamente, ponderação igualitária setorial (\textit{equal-weight}) pode ser considerada, embora \textit{market-cap weighting} seja mais comum e favoreça liquidez.
\item \textit{Rebalanceamento periódico:} Revisão semestral ou anual dos componentes, para incluir eventuais IPOs futuros (ex.: se Creditas ou Ebanx abrirem capital, ou se surgirem novas líderes) e remover empresas que deixem de se qualificar.
\end{itemize}

\textbf{Setorização Interna:} Dentro do ETF, teríamos uma diversificação por subsetores fintech, aproximando-se destas categorias e pesos estimados:
\begin{itemize}
\item Pagamentos e Carteiras Digitais (\textbf{🟦} \textasciitilde30-40\%): Empresas de adquirência, meios de pagamento e \textit{wallets} – ex: PagSeguro, Stone, Visa/Master, PayPal, Mercado Pago (via MELI). Grande peso pois pagamentos é o segmento mais desenvolvido (Pix, cartões, e-commerce).
\item Bancos Digitais e Crédito (\textbf{🟧} \textasciitilde25-35\%): Neobanks e fintechs de empréstimo – ex: Nubank, Inter, Sofi, Upstart. Segmento crucial dado o tamanho do mercado de crédito brasileiro e a disrupção bancária.
\item Investimentos e Seguros (\textbf{🟩} \textasciitilde15\%): Plataformas de investimento, corretoras digitais e \textit{insurtechs} – ex: XP, (eventual) Caixa Seguridade digital, insurtechs inovadoras. Dá exposição ao avanço da democratização financeira.
\item Infraestrutura e B2B (\textbf{🟨} \textasciitilde10-15\%): Fornecedores de tecnologia e soluções B2B – ex: Sinqia, Fiserv, TOTVS, Oracle FSS. Esse bloco traz resiliência e ganha com a expansão do ecossistema como um todo.
\item Outros (\textbf{🟪} \textasciitilde5\%): Fintechs especializadas diversas (por exemplo, \textit{crypto finance} via Coinbase, fintech de gestão financeira pessoal, etc., se incluídas).
\end{itemize}

% --- Placeholder para um gráfico ilustrando a composição do ETF ---
\begin{figure}[!h]
\centering
\includegraphics[width=0.75\textwidth]{composicao_etf_placeholder.png}
\caption{\textbf{Gráfico (Placeholder).} Composição setorial hipotética do ETF Fintech Brasil.}
\label{fig:composicaoETF}
\end{figure}

Essa distribuição assegura que o ETF não fique concentrado demais em um único nicho. Por exemplo, Pagamentos é dominante em número de fintechs (no Brasil \textasciitilde423 startups, 28\% do total) e no volume de investimentos dos últimos anos, mas outros segmentos como Lending (303 startups) e **WealthTech (317 startups)】 também são relevantes. Com a setorização, o investidor do ETF terá uma carteira balanceada representando todo o espectro da revolução fintech.

Vale ressaltar que todas as empresas escolhidas possuem governança e \textit{disclosure} adequados (capital aberto), o que é importante para transparência com os cotistas do ETF. Além disso, ao priorizar liquidez e qualidade, o ETF evita riscos de empresas obscuras ou pouco negociadas, reforçando sua viabilidade operacional.

Por fim, a composição conta a história do fintech brasileiro: desde startups nativas como Nubank e Stone que atraíram investidores globais (Berkshire Hathaway investe em Nuba (Top Brazilian Fintech companies trading on US stock exchanges)6†L28-L36】), até incumbentes transformados e parcerias internacionais. Esse \textit{mix} aumenta a robustez e o apelo do ETF para diversos perfis – tanto investidores locais entusiastas da “prata da casa”, quanto estrangeiros buscando exposição ao promissor mercado brasileiro via um produto estruturado.

\section*{Viabilidade Financeira e Valuation 💰}

Para avaliar a viabilidade do ETF Fintech Brasil, devemos analisar o tamanho de mercado potencial, projeções de crescimento dos ativos e métricas financeiras que sustentem um caso de sucesso para investidores e para a gestora (Investo).

\textbf{Captação Inicial e Mercado Alvo:} Considerando o interesse crescente em ETFs no Brasil – hoje já são mais de 600 mil investidores em ETFs domésticos e internacionais listados na B3 – um fundo temático de fintech tem grande apelo, especialmente junto ao público jovem e conectado, familiar aos serviços de Nubank, PicPay, etc. A Investo, em pouco mais de um ano de operação, captou \textasciitilde R\$ 200 milhões em 12 ETFs (média \textasciitilde R\$16 mi por ETF) e, com aporte recebido, planejava atingir R\$ 1 bi em (Investo capta R\$ 40 milhões para popularizar ETFs no Brasil - Startups) (Investo capta R\$ 40 milhões para popularizar ETFs no Brasil - Startups)†L159-L167】. Com base nisso, estimamos que um ETF de Fintech poderia captar na largada algo entre R\$ 50 e R\$ 100 milhões, dependendo das condições de mercado no lançamento. Esse montante seria viabilizado por alocações de investidores de varejo antenados em tecnologia, \textit{family offices} buscando diversificação temática, e mesmo algum investimento institucional (fundos de fundos, RPPS) que começam a aderir a ETFs setoriais. A título de comparação, a Investo lançou um ETF de cripto (HASH11) em 2021 que captou **\textasciitilde R\$ 30 milhões** rapidamente mesmo em moment (Investo capta R\$ 40 milhões para popularizar ETFs no Brasil - Startups)†L181-L186】 – fintech, sendo um tema menos controverso e mais \textit{mainstream}, tende a atrair \textit{equal} ou maior interesse.

O potencial de mercado a médio prazo é significativo: se considerarmos \textasciitilde50 mil investidores em ETFs na base da Investo (meta (Investo capta R\$ 40 milhões para popularizar ETFs no Brasil - Startups)†L159-L167】, e que fintech é um dos tópicos mais populares, poderíamos ver alguns milhares de cotistas aderindo e o patrimônio crescendo conforme o desempenho for favorável. Além disso, o ETF pode servir como veículo para investidores internacionais acessarem o “Brazil fintech story” via B3, aumentando a base de demanda.

\textbf{Projeções de Crescimento do Setor:} Fundamentar o \textit{pitch} em projeções sólidas é crucial. Conforme já citado, as receitas das fintechs na AL devem crescer \textasciitilde27\% ao ano até 2028, acima de outras regiões, impulsionadas por mercados (O novo cenário das fintechs em 2024, de olho nos mercados B2B – ANBIMA)†L223-L231】. Globalmente, a BCG projeta o **mercado de fintech multiplicando \textasciitilde5x até 2030**, atingindo US\$ 1,5 trilhão e (Relatório aponta o "novo" ecossistema global de fintechs - Finsiders Brasil)†L154-L162】. Isso implica um CAGR de \textasciitilde25\% mundialmente até 2030. Se assumirmos que as empresas do ETF consigam acompanhar algo próximo desses crescimentos de receita e que convertam em lucros à medida que escalam, podemos traçar cenários de retorno atrativos.

Uma modelagem simplificada: suponha que o conjunto de empresas do ETF hoje tenha um lucro agregado de “100” (base hipotética) e esteja precificado a, digamos, 25 vezes lucro (P/L médio). Crescendo o lucro a 20\% a.a. nos próximos 5 anos (conservador frente aos 25\% de receita global), o lucro total seria \textasciitilde249 em cinco anos (crescimento \textasciitilde2,5x). Mesmo assumindo uma contração de múltiplo para 20x lucro (investidores pagando relativamente menos conforme setor amadurece), o valor agregado das empresas seria 249$\times$20 = 4.980, comparado a 100$\times$25 = 2.500 hoje. Isso representa praticamente dobrar o valor de mercado em 5 anos, o que resultaria num retorno anualizado próximo de 15\%. Esse seria o cenário base/otimista para apresentar: investidores poderiam ver seu capital duplicar em \textasciitilde5 anos acompanhando a expansão do setor fintech. Em um cenário mais otimista (crescimento >20\% a.a. ou manutenção de múltiplos altos dada a inovação contínua), o retorno poderia ser ainda maior. Por exemplo, se as fintechs brasileiras crescerem receitas \textasciitilde3x mais rápido que bancos (com (Fintechs deverão crescer dois dígitos ao ano até 2028 | McKinsey)67†L42-L46】 e alcançarem parte do valuation dos incumbentes, o \textit{upside} é expressivo – lembremos que o setor bancário tradicional brasileiro ainda é avaliado em centenas de bilhões de reais de \textit{market cap}, espaço que as fintechs podem abocanhar.

Por outro lado, é importante tratar dos riscos e cenários conservadores. No \textit{worst case}, caso a adoção de fintechs desacelere ou haja forte concorrência derrubando margens, poderíamos ver crescimento de lucros mais modesto (ex.: 5-10\% a.a.) ou até fracassos de uma ou outra empresa. No entanto, a estrutura diversificada do ETF mitiga o risco de uma empresa específica: por exemplo, mesmo que uma fintech enfrente dificuldades (digamos, aumento de inadimplência em uma fintech de crédito), outras de pagamentos ou software podem compensar. Adicionalmente, os resultados recentes mostram resiliência – em meio a cenário de juros altos em 2022/23, as fintechs brasileiras melhoraram seus lucros em +38\% e ajust (Sete de oito fintechs brasileiras registram lucro no primeiro semestre -) (Sete de oito fintechs brasileiras registram lucro no primeiro semestre -)†L146-L154】, indicando capacidade de adaptação. Isso dá confiança nas projeções de longo prazo.

\textbf{Métricas Fundamentais para o Investidor:} Para tornar o \textit{pitch} impactante, podemos destacar métricas financeiras agregadas do ETF, comparando-as com \textit{benchmarks}. Por exemplo:
\begin{itemize}
\item Crescimento de receitas (últimos 3 anos) das componentes do ETF vs. grandes bancos: (hipotético) +30\% a.a. vs +5\% a.a. – ilustra a disrupção.
\item ROE médio ou margem bruta das fintechs vs bancos incumbentes: muitas fintechs têm margens brutas altas (>50\%) devido à natureza digital, embora reinvistam muito, ao passo que bancos têm ROEs altos mas com modelos diferentes. Poderia-se mostrar que conforme fintechs escalonarem, podem atingir rentabilidades semelhantes.
\item Índice Preço/Valor Patrimonial: bancos tradicionais no Brasil negociam em torno de 1,5-2,0x PV, enquanto fintechs frequentemente estão acima disso mas justificadamente pelo crescimento. Apontar que o P/VPA médio do ETF talvez fiq (FinTech ETF (FINX))†L268-L272】 (estimativa baseada no FINX) – indicando prêmio de inovação, porém não exorbitante.
\item Relação LTV/CAC: métrica de \textit{unit economics} – McKinsey cita que 68\% das fintechs na LatAm têm LTV/CAC > 5 (ou seja, cada cliente vale 5x o custo de (Fintechs deverão crescer dois dígitos ao ano até 2028 | McKinsey)67†L65-L72】, demonstrando sustentabilidade dos modelos de negócio que compõem o ETF.
\item \textit{Sharpe Ratio} esperado: Se projetamos \textasciitilde15\% de retorno anual com uma volatilidade (desvio) \textasciitilde30\%, o Sharpe (excesso de retorno / risco) pode ser similar ao de um índice de ações, mas com potencial de melhora conforme o crescimento se materializa e a volatilidade tende a reduzir quando empresas ficam maiores.
\end{itemize}

\textbf{Viabilidade Operacional:} Do ponto de vista técnico, montar o ETF é viável via investimento nos ativos domésticos e BDRs para os estrangeiros. A B3 permite composição de ETF com até 100\% estrangeiros via \textit{feeder} (como já feito nos ETFs Investo que replicam índices externos). No caso, poderíamos replicar um índice criado \textit{ad hoc} (ex.: Índice Investo Fintech Brasil) e ter o ETF listado possivelmente com \textit{ticker} FINT11 ou semelhante. A liquidez das ações subjacentes é suficiente para suportar criação e resgate de cotas. A taxa de administração poderia ser na faixa de 0,7\% a.a., gerando receita para a gestora e ainda assim competitiva dado o nicho. Com AUM de R\$ 100 milhões, isso seria R\$ 700 mil/ano de receita bruta, cobrindo custos de licenciamento do índice e operação.

Do ponto de vista do investidor, a tese de investimento é clara e empolgante: participar do crescimento da “Nova Finança” no Brasil, diversificando em empresas líderes do futuro financeiro. Muitos usuários de fintechs possivelmente se tornariam investidores do ETF para “apostar” no sucesso das marcas que eles próprios utilizam – há um elemento de conexão emocional e de marketing aí (similar ao que houve com BIDI11 e o Nubank IPO atraindo pessoas físicas). Esse engajamento do público final melhora a viabilidade comercial do produto.

\textbf{Resumo do Caso:} O ETF Fintech Brasil une forte narrativa + fundamentos concretos. De um lado, temos histórias de sucesso (Nubank, Stone, XP) e dados de mercado (milhões de usuários, bilhões transacionados via Pix, etc.) que tornam o \textit{pitch} tangível. De outro, temos indicadores financeiros robustos: crescimento de dois dígitos, empresas caminhando para lucratividade sólida (7 de 8 já (Sete de oito fintechs brasileiras registram lucro no primeiro semestre -)†L124-L132】), mercado endereçável gigantesco (setor financeiro brasileiro) e apoio regulatório contínuo. Essa combinação sustenta a viabilidade financeira do ETF, tanto em atrair capital inicialmente quanto em entregar retornos competitivos aos investidores no médio/longo prazo.

Em essência, investir no ETF Fintech Brasil é apostar na modernização financeira do país – um movimento que já está em curso e tende a acelerar nos próximos anos, trazendo ganhos para quem entrar nessa jornada desde já.

\section*{Visualizações e Gráficos 📈🖥️}

Para fortalecer o \textit{pitch}, é importante apresentar visualizações claras que ilustrem os pontos-chave. Algumas sugestões de gráficos e como eles embasam a narrativa:

\begin{itemize}
\item \textbf{Comparativo do Ecossistema Fintech (Brasil vs. Mundo):} Um gráfico de barras ou mapa pode mostrar o número de fintechs no Brasil em comparação a outros hubs globais. Por exemplo, destacar que o Brasil (com \textasciitilde1500 fintechs) está à frente de mercados emergentes e é comparável a países desenvolvidos. Também um gráfico pizza da distribuição de fintechs na LatAm: Brasil sozinho tem 31\% das fin (O novo cenário das fintechs em 2024, de olho nos mercados B2B – ANBIMA). Isso visualmente reforça a liderança brasileira. Além disso, poderia incluir um gráfico de financiamento acumulado 2014-2023 mostrando os \textasciitilde US\$10 bi investidos no Brasil vs \textasciitilde US\$5 bi no r (Fintechs brasileiras atraíram quase R\$ 60 bi em investimentos em 10 anos, diz estudo sobre as startups que desafiam os bancos | Startups | Época NEGÓCIOS) (Fintechs brasileiras receberam 66\% dos aportes de LatAm em 10 anos - Startups)30†L139-L147】 – evidenciando a concentração de recursos aqui.

\item \textbf{Gráfico de Tendência de Investimentos:} Plotar uma linha temporal de \textit{funding} em fintech por ano (2016 a 2023, por exemplo) evidenciando o pico de 2021 e a correção de 2022. Isso contextualiza o momento atual: apesar do ajuste pós-pico, a tendência de longo prazo permanece positiva. Poderíamos sobrepor duas linhas: Fintech Brasil vs. Fintech Global – ambas subiram muito até 2021 e depois caíram \textasciitilde5 (Relatório aponta o "novo" ecossistema global de fintechs - Finsiders Brasil)27†L124-L132】, indicando que foi um movimento setorial global, não algo errado com Brasil. Em 2023 já se nota recuperação (poder marcar ponto H1 2024 +7 (Fintechs brasileiras receberam 66\% dos aportes de LatAm em 10 anos - Startups)30†L161-L169】). Essa visualização tranquiliza investidores quanto à volatilidade passageira de investimentos.

\item \textbf{Composição do ETF – Setores e Empresas:} Aqui um gráfico em pizza seria didático, mostrando as fatias aproximadas do ETF por categoria: ex.: 35\% Pagamentos, 30\% Bancos Digitais/Crédito, 20\% Investimentos/Wealth, 15\% Infraestrutura. Cada fatia com cor e talvez logos das principais empresas daquele segmento. Isso permitiria ao investidor “enxergar” a diversificação interna. Alternativamente, um diagrama de árvore com os Top 10 ativos e seus pesos: ex.: Nubank 15\%, PagSeguro 10\%, Stone 10\%, XP 10\%, MercadoLibre 8\%, Inter 8\%, etc (apenas ilustrativo). Esse tipo de gráfico mostra quem são as estrelas do portfólio e evita qualquer confusão de que o ETF seja concentrado demais – visualmente veriam vários nomes relevantes. Podemos também incluir mini-logos das empresas para identificação imediata (no material de apresentação, isso costuma causar impacto visual).

\item \textbf{Gráfico de Performance Simulada/Benchmark:} Para convencer sobre retornos, um gráfico comparando o desempenho de um índice fintech vs. Ibovespa nos últimos anos seria poderoso. Por exemplo, mostrar um índice hipotético composto por Nubank, Stone, PagSeguro, XP desde Jan/2018 vs. IBOV. Provavelmente, apesar da queda recente, esse índice fintech ainda superaria o IBOV (que ficou de lado nesse período). Se houver dados, podemos usar o ETF FINX vs. S\&P500 como proxy: no acumulado desde 2016, FINX +118\% vs S\&P500 \textasciitilde+10 (FinTech ETF (FINX)). Colocar duas linhas mostrando a fintech vencendo no longo prazo reforça a ideia de “\textit{alpha}”. Também dá para plotar a volatilidade – a linha fintech oscila mais – e explicar que isso é parte do pacote de maior retorno.

\item \textbf{Projeção de Crescimento:} Por fim, um gráfico projetivo para ilustrar aquela modelagem de valor futuro. Exemplo: um gráfico de área mostrando o tamanho de mercado das fintechs crescendo até 2030. Poderia marcar: 2023 – fintech revenue \textasciitilde\$320 bi, 20 (Relatório aponta o "novo" ecossistema global de fintechs - Finsiders Brasil)29†L154-L162】, praticamente uma parede ascendente de crescimento. Isso comunica visualmente o enorme espaço a ser conquistado. Outra ideia é um gráfico de \textit{market share}: fintechs hoje <5\% das receitas financeiras, projetado (Relatório aponta o "novo" ecossistema global de fintechs - Finsiders Brasil) (Relatório aponta o "novo" ecossistema global de fintechs - Finsiders Brasil)29†L154-L162】, indicando que quem investir cedo poderá capturar essa transição estrutural.
\end{itemize}

% --- Placeholder para um gráfico de performance simulada ---
\begin{figure}[!h]
\centering
\includegraphics[width=0.75\textwidth]{performance_simulada_placeholder.png}
\caption{\textbf{Gráfico (Placeholder).} Exemplo de comparação de performance simulada Fintech vs. Ibovespa.}
\label{fig:performanceSim}
\end{figure}

% --- Placeholder para um gráfico de projeção de crescimento ---
\begin{figure}[!h]
\centering
\includegraphics[width=0.75\textwidth]{projecao_crescimento_placeholder.png}
\caption{\textbf{Gráfico (Placeholder).} Projeção de crescimento do setor Fintech até 2030.}
\label{fig:growthProj}
\end{figure}

Em complemento aos gráficos, infográficos podem ser usados para fatos-chave: por exemplo, um ícone de banco vs. celular com números de clientes – “Nubank 100+ milhões de clientes (já (Nubank se torna o 3º maior banco do Brasil em número de clientes))”, “Pix: 42 bi transações em 2023 (75\% (FEBRABAN - Notícias)”, “Fintechs na AL crescerão 27\% a.a (2023-28) (O novo cenário das fintechs em 2024, de olho nos mercados B2B – ANBIMA)”, etc. Esses destaques visuais fixam na memória do investidor os motivos para acreditar no tema.

Em suma, os gráficos e visualizações irão transformar dados em uma história visualmente atraente, reforçando os pontos do \textit{pitch}: o Brasil lidera uma revolução fintech (gráfico de participação), o setor está em forte expansão (linhas de investimento/crescimento), o ETF será diversificado e bem construído (pizza de composição) e o potencial de retorno é significativo (linhas de performance/projeção). Esses elementos visuais combinados com a análise fundamentada fornecerão à Investo um \textit{pitch} memorável e convincente para lançar o ETF Fintech Brasil com sucesso.

\end{document}
